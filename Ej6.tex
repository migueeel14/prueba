\rule[2ex]{\textwidth}{2pt}\\


\begin{enumerate}
    \item[6.] Resuelva el PVI
\end{enumerate}
\begin{align*}
    x\dv{y}{x} = -2xy+ \frac{x}{1+e^{2x}}, \qquad y(1)= 1
\end{align*}
\begin{align*}
    4x \dv{y}{x} = -8y + \frac{x}{1+x^{3}}, \qquad y(1)=1
\end{align*}

\begin{itemize}
    \item \begin{proof}
Consideramos la ecuación

$$x\frac{dy}{dx}=-2xy+\frac{x}{1+e^{2x}}$$

sumando $2xy$ en ambos miembros obtenemos

$$x\frac{dy}{dx}+2xy=\frac{x}{1+e^{2x}}$$

dividimos ambos miembros de la ecuación por $x$

$$\frac{dy}{dx}+2y=\frac{1}{1+e^{2x}}$$

aqui observamos que tenemos una ecuación de la forma $\frac{dy}{dx}+p(x)y=q(x)$, por tanto vamos a obtener el factor integrante 

$$\mu=e^{{\int} p(x)dx}$$

para $\int p(x)dx$ tenemos

$$\int p(x)dx=\int 2 dx=2x$$

por tanto tenemos que el factor integrante es

$$\mu=e^{2x}$$

Ahora multiplicamos ambos miembros de la ecuación por $\mu=e^{2x}$ para obtener

$$e^{2x}(\frac{dy}{dx}+2y)=e^{2x}\frac{1}{1+e^{2x}} $$

$$e^{2x}\frac{dy}{dx}+e^{2x}2y=e^{2x}\frac{1}{1+e^{2x}} $$

el miembro de la izquierda representa la diferencial del producto de la función a buscar $y(x)$ con la función $e^{2x}$, es decir:

$$d(y\cdot e^{2x})=e^{2x}\frac{1}{1+e^{2x}}$$

integramos ambos miembros de la ecuación anterior 

$$\int d(y\cdot e^{2x})dx=\int e^{2x}\frac{1}{1+e^{2x}}dx$$

$$y\cdot e^{2x}=\int e^{2x}\frac{1}{1+e^{2x}}dx$$

por lo que obtenemos

$$y=\frac{1}{e^{2x}}\int \frac{e^{2x}}{1+e^{2x}}dx$$

resolvemos la integral tomando $u=1+e^{2x}$, obtenemos

$$y=\frac{1}{e^{2x}}\left[ln(e^{2x}+1)^{\frac{1}{2}}+C\right]$$

ahora sustituyendo los valores iniciales $y(1)=1$ para obtener la constante 

$$1=\frac{1}{e^2}\left[ln(e^2+1)^{\frac{1}{2}}+C\right]$$

despejando C obtenemos

$$C=e^2-ln(e^2+1)^{\frac{1}{2}}$$

sustituyendo obtenemos

$$y=\frac{1}{e^{2x}}\left[ln(e^{2x}+1)^{\frac{1}{2}}+e^2-ln(e^2+1)^{\frac{1}{2}}\right]$$

por tanto la solución al PVI es

$$y=\frac{e^2}{e^{2x}}=e^{2(1-x)}$$

\end{proof}

  \item \begin{proof}

Consideramos la ecuación

$$4x\frac{dy}{dx}=-8y+\frac{x}{1+x^3}$$

sumando $8y$ en ambos miembros obtenemos

$$4x\frac{dy}{dx}+8y=\frac{x}{1+x^3}$$

dividimos ambos miembros de la ecuación por $4x$

$$\frac{dy}{dx}+\frac{2y}{x}=\frac{1}{4(1+x^3)}$$

aqui observamos que tenemos una ecuación de la forma $\frac{dy}{dx}+p(x)y=q(x)$, por tanto vamos a obtener el factor integrante 

$$\mu=e^{{\int} p(x)dx}$$

para $\int p(x)dx$ tenemos

$$\int p(x)dx=\int \frac{2}{x} dx=2ln(x)$$

por tanto tenemos que el factor integrante es

$$\mu=e^{2ln(x)}=x^2$$

Ahora multiplicamos ambos miembros de la ecuación por $\mu=x^{2}$ para obtener

$$x^2(\frac{dy}{dx}+\frac{2y}{x})=x^2\frac{1}{1+x^3} $$

$$x^2\frac{dy}{dx}+x^2\frac{2y}{x}=x^2\frac{1}{1+x^3}$$

el miembro de la izquierda representa la diferencial del producto de la función a buscar $y(x)$ con la función $x^2$, es decir:

$$d(y\cdot x^2)=x^2\frac{1}{1+x^3}$$

integramos ambos miembros de la ecuación anterior 

$$\int d(y\cdot x^2)dx=\int x^2\frac{1}{1+x^3}dx$$

$$y\cdot x^2=\int \frac{x^2}{1+x^3}dx$$

por lo que obtenemos

$$y=\frac{1}{x^2}\int \frac{x^2}{1+x^3}dx$$

resolvemos la integral tomando $u=1+x^3$, obtenemos

$$y=\frac{1}{x^2}\left[ln|x^3+1|^{\frac{1}{3}}+C\right]$$

ahora sustituyendo los valores iniciales $y(1)=1$ para obtener la constante 

$$1=\frac{1}{1}\left[ln(2)^{\frac{1}{3}}+C\right]$$

despejando C obtenemos

$$C=-ln(2)^{\frac{1}{3}}$$

sustituyendo obtenemos

$$y=\frac{1}{x^2}\left[ln|x^3+1|^{\frac{1}{3}}-ln(2)^{\frac{1}{3}}\right]$$

por tanto la solución al PVI es

$$y=x^{-2}\left[ ln|\frac{x^3+1}{2}|^{\frac{1}{3}} \right]$$
  
  
\end{proof}

 
\end{itemize}

\rule[2ex]{\textwidth}{2pt}\\





