\begin{enumerate}
    \item Demuestre que las siguientes ecuaciones no son exactas, encuentra el factor integrante que las vuelve exactas y resuelve los PVI.
\begin{enumerate}
    \item $(10-6y+e^{-3x})\dd x- 2 \dd y=0, \qquad y(0)=1$
    
\textit{ \textbf{Solucion}}
    
    \begin{align*}
        M(x,y) &= 10-6y+e^{-3x}\\
        N(x,y) &=-2
\end{align*}
Notemos que $\displaystyle \frac{\partial M}{\partial y} \neq \frac{\partial N}{\partial x}$, por lo tanto, busquemos un factor integrante $\mu$ que la vuelva exacta.
\begin{align*}
    \frac{\partial M}{\partial y} = \frac{\partial}{\partial y}(10-6y+e^{3x})&=-6\\
    \frac{\partial N}{\partial x} = \frac{\partial}{\partial x} (-2)&=0\\
    \frac{\frac{\partial M}{\partial y} - \frac{\partial N}{\partial x}}{-2} = \frac{-6}{-2} &= 3
\end{align*}
que en particular depende de $x$, Así
\begin{align*}
    \mu(x)= e^{\int 3dx}\implies e^{3x}
\end{align*}
entonces
\begin{align*}
    e^{3x}[(10-6y+e^{-3x})dx-2dy]&=0\\
    e^{3x}(10-6y+e^{-3x})dx-e^{3x}2ydy & = 0 \numberthis
\end{align*}
Comprobemos que es exacta
\begin{align*}
    M_{y}=e^{3x}(-6), \qquad & N_{x}= -3e^{3x}(2)\\
    M_{y}=-6e^{3x}, \qquad & N_{x}= -6e^{3x}
\end{align*}
Por lo tanto (1) es exacta.\\
Calculemos $\displaystyle \int M dx$
\allowdisplaybreaks
\begin{align*}
    \int (10-e^{3x}-6ye^{3x}+e^{0}) &=\int 10e^{3x}-\int 6ye^{3x}dx+\int dx\\
    & = 10 \int e^{3x} - 6y\int e^{3x}dx+\int dx\\
    & = \frac{10}{3}e^{3x}-2ye^{3x}+x+h'(y) = f(x,y)\\
    \frac{\partial f}{\partial y} & = - 2e^{3x} + h'(y) = N(x,y)\\
    \implies h'(y) & = -2e^{3x} + 2e^{3x} =0\\
    \therefore \qquad h(y)& = C
\end{align*}
por lo tanto
\begin{align*}
    \frac{10}{3} e^{3x}  - 2ye^{3x} + x = C_{1} 
\end{align*}
Resolviendo el PVI 
\begin{align*}
    c &= \frac{10}{3} e^{3x} -2(1)e^{3x} +0 \\ 
    c &= \frac{4}{3}
\end{align*}
    \item $\cos x \dd x+ \left( 1 + \frac{2}{y} \right) \sen x \dd y=0, \qquad y(0)=2$
    
\textit{ \textbf{Solucion}}
    
\begin{align*}
        M(x,y)&=\cos(x) \\
        N(x,y)&=1+\frac{2}{y} \sen(x) 
\end{align*}
\begin{align*}
    \frac{\partial M}{\partial y} = 0 \neq \left( 1+\frac{2}{y}\right) \cos (x) = \frac{\partial N}{\partial x}
\end{align*}
Encontrar el factor integrante
\begin{align*}
\frac{N_{x}-M_{y}}{M} = \frac{\left( 1+\frac{2}{y}\right) \cos (x)}{\cos(x)}=1 +\frac{2}{y}
\end{align*}
Para obtener el factor integrante
\begin{align*}
    \mu(x) = e^{\int \left( 1 + \frac{2}{4}\right)dy}=e^{y+2\ln\abs{y}} = e^{y}y^{2}
\end{align*}
multiplicando EDO por $\mu(x)$
\begin{align*}
    e^{y}y^{2} \cos(x)dx + \left( 1+\frac{2}{y} \right) e^{y}y^{2}\sen(x)dy = 0
\end{align*}
\begin{align*}
    \frac{\partial M}{\partial y} &= \cos (x) e^{y}y(y+2)\\
    \frac{\partial N}{\partial x} &= \cos (x)\left( 1+\frac{2}{y} \right) e^{y}y^{2}= \cos(x)e^{y}y(y+2)\\
\end{align*}
por lo tanto $\quad \displaystyle \frac{\partial M}{\partial y} = \frac{\partial N}{\partial x} $
\begin{align*}
    \int M dx =\int e^{y}y^{2}\cos(x)dx&=e^{y}y^{2}\sen(x)h(y)\\
    \frac{\partial}{\partial y} (e^{y}y^{2}\sen(x)+h(y)) & = \sen(x)e^{y}y(y+2)+h(y)
\end{align*}
pero
\begin{align*}
    \sen(x)e^{y}y(y+2)+h'(x) &=e^{y}y(+2)\sen(x)\\
    \therefore h'(x)&=0\\
    \therefore h (x)& =c
\end{align*}
Por lo tanto 
\begin{align*}
    f(x,y) &= e^{y}y^{2}\sen(x)+c\\
    c &= e^{y}y^{2}\sen(x)
\end{align*}
Resolviendo el PVI $y(0)=2$
\begin{align*}
    c & = e^{y}y^{2}\sen(0)=0 \\
    c & = 0
\end{align*}
\end{enumerate}
\end{enumerate}